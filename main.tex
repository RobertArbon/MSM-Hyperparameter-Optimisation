
%%%%%%%%%%%%%%%%%%%%%%%%%%%%%%%%%%%%%%%%%%%%%%%%%%%%%%%%%%%%%%%%%%%%%
%% This is a (brief) model paper using the achemso class
%% The document class accepts keyval options, which should include
%% the target journal and optionally the manuscript type. 
%%%%%%%%%%%%%%%%%%%%%%%%%%%%%%%%%%%%%%%%%%%%%%%%%%%%%%%%%%%%%%%%%%%%%
\documentclass[journal=jacsat,manuscript=article]{achemso}

%%%%%%%%%%%%%%%%%%%%%%%%%%%%%%%%%%%%%%%%%%%%%%%%%%%%%%%%%%%%%%%%%%%%%
%% Place any additional packages needed here. Only include packages
%% which are essential, to avoid problems later. Do NOT use any
%% packages which require e-TeX (for example etoolbox): the e-TeX
%% extensions are not currently available on the ACS conversion
%% servers.
%%%%%%%%%%%%%%%%%%%%%%%%%%%%%%%%%%%%%%%%%%%%%%%%%%%%%%%%%%%%%%%%%%%%%
\usepackage[version=3]{mhchem} % Formula subscripts using \ce{}
\usepackage{siunitx}
\usepackage{tabularx}
\usepackage{float}
\usepackage{booktabs}
\usepackage{subcaption}
\usepackage{amsmath}
\usepackage{amssymb}
\usepackage{amsfonts}
\usepackage{bm}
\usepackage{xfrac}
\usepackage{graphicx}
\DeclareMathOperator*{\argmax}{arg\,max}
\DeclareMathOperator*{\argmin}{arg\,min}
\newcommand{\SIci}[4]{\SI{#1}{#4},\ \SI{95}{\percent}C.I.\ [\numrange[range-phrase=---]{#2}{#3} \si{#4}]}
\newcommand{\numci}[3]{\num{#1},\ \SI{95}{\percent}C.I.\ [\numrange[range-phrase=---]{#2}{#3}]}
% Fancy table stuff
\newcommand{\nextitem}{\par\hspace*{\labelsep}\textbullet\hspace*{\labelsep}}
\newcolumntype{Z}{>{\centering\let\newline\\\arraybackslash\hspace{0pt}}X}

% Shorthand for features
\newcommand{\distlabel}{$dist.\ $}
\newcommand{\logitdistlabel}{$\mathrm{logit}(dist.)\ $}
\newcommand{\dihedlabel}{$dihed.\ $}

%%%%%%%%%%%%%%%%%%%%%%%%%%%%%%%%%%%%%%%%%%%%%%%%%%%%%%%%%%%%%%%%%%%%%
% supplementary materials
\usepackage{xr}
\newcommand*\sref[1]{%
    S\ref{#1}}
    
\makeatletter
\newcommand*{\addFileDependency}[1]{% argument=file name and extension
  \typeout{(#1)}
  \@addtofilelist{#1}
  \IfFileExists{#1}{}{\typeout{No file #1.}}
}
\makeatother

\newcommand*{\myexternaldocument}[1]{%
    \externaldocument{#1}%
    \addFileDependency{#1.tex}%
    \addFileDependency{#1.aux}%
    }
    
\myexternaldocument{SI}
%%%%%%%%%%%%%%%%%%%%%%%%%%%%%%%%%%%%%%%%%%%%%%%%%%%%%%%%%%%%%%%%%%%%%

% \usepackage{caption}
% \captionsetup[table]{position=bottom} 
% \usepackage{subcaption}
% \usepackage{bm} % e.g., \bm(\mu)
% \usepackage{xfrac}  % e.g., \sfrac{1}{2}                   
% \usepackage{relsize} % e.g., \mathlarger 
% \usepackage{algorithm2e}
% \DeclareMathOperator*{\argmax}{arg\,max}
% \DeclareMathOperator*{\argmin}{arg\,min}

%%%%%%%%%%%%%%%%%%%%%%%%%%%%%%%%%%%%%%%%%%%%%%%%%%%%%%%%%%%%%%%%%%%%%
%% If issues arise when submitting your manuscript, you may want to
%% un-comment the next line. This provides information on the
%% version of every file you have used.
%%%%%%%%%%%%%%%%%%%%%%%%%%%%%%%%%%%%%%%%%%%%%%%%%%%%%%%%%%%%%%%%%%%%%
%%\listfiles

%%%%%%%%%%%%%%%%%%%%%%%%%%%%%%%%%%%%%%%%%%%%%%%%%%%%%%%%%%%%%%%%%%%%%
%% Place any additional macros here. Please use \newcommand* where
%% possible, and avoid layout-changing macros (which are not used
%% when typesetting).
%%%%%%%%%%%%%%%%%%%%%%%%%%%%%%%%%%%%%%%%%%%%%%%%%%%%%%%%%%%%%%%%%%%%%
\newcommand*\mycommand[1]{\texttt{\emph{#1}}}

%%%%%%%%%%%%%%%%%%%%%%%%%%%%%%%%%%%%%%%%%%%%%%%%%%%%%%%%%%%%%%%%%%%%%
%% Meta-data block
%% ---------------
%% Each author should be given as a separate \author command.
%%
%% Corresponding authors should have an e-mail given after the author
%% name as an \email command. Phone and fax numbers can be given
%% using \phone and \fax, respectively; this information is optional.
%%
%% The affiliation of authors is given after the authors; each
%% \affiliation command applies to all preceding authors not already
%% assigned an affiliation.
%%
%% The affiliation takes an option argument for the short name. This
%% will typically be something like "University of Somewhere".
%%
%% The \altaffiliation macro should be used for new address, etc.
%% On the other hand, \alsoaffiliation is used on a per author basis
%% when authors are associated with multiple institutions.
%%%%%%%%%%%%%%%%%%%%%%%%%%%%%%%%%%%%%%%%%%%%%%%%%%%%%%%%%%%%%%%%%%%%%

\author{Robert E. Arbon}
\altaffiliation{ReDesign Science, New York, NY, USA}
\author{Antonia S.J.S. Mey}
\email{antonia.mey@ed.ac.uk}
\affiliation[Unknown University]
{EaStCHEM School of Chemistry, David Brewster Road, Joseph Black Building, The King’s Buildings, Edinburgh, EH93FJ, UK}

%%%%%%%%%%%%%%%%%%%%%%%%%%%%%%%%%%%%%%%%%%%%%%%%%%%%%%%%%%%%%%%%%%%%%
%% The document title should be given as usual. Some journals require
%% a running title from the author: this should be supplied as an
%% optional argument to \title.
%%%%%%%%%%%%%%%%%%%%%%%%%%%%%%%%%%%%%%%%%%%%%%%%%%%%%%%%%%%%%%%%%%%%%
\title[]{Sensitivity tests for automated Markov state modelling}

%%%%%%%%%%%%%%%%%%%%%%%%%%%%%%%%%%%%%%%%%%%%%%%%%%%%%%%%%%%%%%%%%%%%%
%% Some journals require a list of abbreviations or keywords to be
%% supplied. These should be set up here, and will be printed after
%% the title and author information, if needed.
%%%%%%%%%%%%%%%%%%%%%%%%%%%%%%%%%%%%%%%%%%%%%%%%%%%%%%%%%%%%%%%%%%%%%
\abbreviations{IR,NMR,UV}
\keywords{American Chemical Society, \LaTeX}

%%%%%%%%%%%%%%%%%%%%%%%%%%%%%%%%%%%%%%%%%%%%%%%%%%%%%%%%%%%%%%%%%%%%%
%% The manuscript does not need to include \maketitle, which is
%% executed automatically.
%%%%%%%%%%%%%%%%%%%%%%%%%%%%%%%%%%%%%%%%%%%%%%%%%%%%%%%%%%%%%%%%%%%%%
\begin{document}

%%%%%%%%%%%%%%%%%%%%%%%%%%%%%%%%%%%%%%%%%%%%%%%%%%%%%%%%%%%%%%%%%%%%%
%% The "tocentry" environment can be used to create an entry for the
%% graphical table of contents. It is given here as some journals
%% require that it is printed as part of the abstract page. It will
%% be automatically moved as appropriate.
%%%%%%%%%%%%%%%%%%%%%%%%%%%%%%%%%%%%%%%%%%%%%%%%%%%%%%%%%%%%%%%%%%%%%
\begin{tocentry}

Some journals require a graphical entry for the Table of Contents.
This should be laid out ``print ready'' so that the sizing of the
text is correct.

Inside the \texttt{tocentry} environment, the font used is Helvetica
8\,pt, as required by \emph{Journal of the American Chemical
Society}.

The surrounding frame is 9\,cm by 3.5\,cm, which is the maximum
permitted for  \emph{Journal of the American Chemical Society}
graphical table of content entries. The box will not resize if the
content is too big: instead it will overflow the edge of the box.

This box and the associated title will always be printed on a
separate page at the end of the document.

\end{tocentry}

%%%%%%%%%%%%%%%%%%%%%%%%%%%%%%%%%%%%%%%%%%%%%%%%%%%%%%%%%%%%%%%%%%%%%
%% The abstract environment will automatically gobble the contents
%% if an abstract is not used by the target journal.
%%%%%%%%%%%%%%%%%%%%%%%%%%%%%%%%%%%%%%%%%%%%%%%%%%%%%%%%%%%%%%%%%%%%%
\begin{abstract}
  Abstract
\end{abstract}

%%%%%%%%%%%%%%%%%%%%%%%%%%%%%%%%%%%%%%%%%%%%%%%%%%%%%%%%%%%%%%%%%%%%%
%% Start the main part of the manuscript here.
%%%%%%%%%%%%%%%%%%%%%%%%%%%%%%%%%%%%%%%%%%%%%%%%%%%%%%%%%%%%%%%%%%%%%
\section{Introduction}


Markov state models (MSMs) are a popular model for extracting kinetic information from unbiased molecular dynamics simulations. Studies published in the last two years alone include a wide range of applications, such as understanding:  protein association kinetics [recent noe/olsson nature paper, Cannariato], Enzyme dynamics [koulgi], ion binding mechanisms  [dutta, McKiernan] , hydrogen bond dynamics [ibrahim], mechanisms of drug binding for drug discovery [hu, Pantsar, Hempel, Tosstorff, Liu, ...], mutational effects conformational dynamics [Fernandez-Quintero, Sharma], kinetics of intrinsically disordered proteins [Paul], protein folding [Zhou], understanding allostery [Tian].  Estimating an MSM proceeds by first a dataset of unbiased molecular dynamics (MD) simulation data, then associating each molecular configurations with a set of discrete states, counting transitions between the states separated by the temporal resolution of the model ($\tau$), and then deriving transition probabilities between the states~\cite{trendelkamp-schroer_estimation_2015}. The final model is summarised by the transition matrix, $\mathbf{T}$ where the elements, $T_{i, j}$ are the conditional probabilities of being in state $i$ at time $t$ and then transitioning to a state $j$ at a time $t+\tau$ : $T_{i,j}(\tau) = P(j, t=t+\tau| i, t=t)$.  

The whole pipeline of transforming MD frames into into a transition matrix involves a making number of modelling choices also called \emph{hyperparameters}. Hyperparameters are differentiated from the \emph{parameters} of the model because the latter are calculated from the data via the optimisation of a loss-function (e.g., the log-likelihood), while the hyperparameters are chosen via expert judgement, or via some summary metric of the model. To take a simple example, when faced with predicting some outcome $y$ against an number of different features $x_i$ as a linear model $y = \beta_0  + \beta_1 x_1, + \beta_2 x_2 ..$, the parameters $\beta_i$ are determined from the data by maximizing the log likelihood, while the choice of whether or not to include $x_3$ or not in the definition of the model may be taken by looking at the accuracy of the model.  For MSMs, the more important hyperparameters are which subset of atoms of simulation to include (typically these are the atomic positions of the solutes); how to transform these coordinates into important  features (e.g., contact maps may be useful for describing protein folding); estimating and projecting onto important collective variables (typically time-lagged independent component analysis, TICA [pande] is used for this process); and finally how to define discrete states from these collective variables.  

Hyperparameter optimisation is a widely studied topic in machine learning [refs from thesis] where many hyperparameters (e.g., learning rates and schedules, network architecture, regularization etc. in deep learning) need to be carefully chosen to achieve the best performance. Various methods exist finding the optimal set of hyperparameters, from exhaustively searching over a uniformly space grid or randomly selected from a predefined search space [bergstra], to active learning approaches such as Bayesian optimisation [from thesis]. 

While MSMs do not typically have a `ground truth' comparison available, variational scores have allowed them to be amenable to hyperparameter optimisation and model selection. The first to be developed was the cross-validated generalized matrix Rayleigh quotient~\cite{mcgibbonVariationalCrossvalidationSlow2015}, GRMQ, which pertains to reversible MSMs; while the variational approach to Markov processes (VAMP) scores~\cite{ wuVariationalApproachLearning2020c, scherer_variational_2019} extended these ideas to both reversible, non-reversible and non-stationary models. 

Hyperparameter optimisation is part of the larger discipline of model selection [?] where the researcher must select and justify their particularly analytic approach. Opaque model selection has lead some to argue [science AI repro paper] that artificial intelligence and machine learning is undergoing a `reproducibility crisis' akin to the widely discussed `crisis' in life-sciences such as psychology and neuroscience [munafo nature]. This crisis manifests in eye-catching results not being replicable by outside researchers []. Part of the cause is that researchers are not transparent with hyperparameter and model selection. 
One class of approaches  suggested to improve model selection  is \emph{sensitivity analysis}, where modelling choices are varied and their effect on model observables is analysed [uncertainty analysis book ref]. Other methods such as multiverse analysis [], vibration of effects [] and specification curve analysis [], take a similar approach whereby large numbers of plausible models are estimated and inferences drawn from the ensemble of results. 

Our reading of the recent literature on applications of MSMs suggest that justifying and reporting of hyperparameter selection using variational scores or other methods is rare[refs from excel].  While we do not suggest that this constitutes a crisis, it does suggest that more can be done to promote good model selection practice and reporting. 

To address this issue this work recommends a set of best practices for optimizing and reporting MSMs. Many of these recommendations can be, with suitable adjustment, be generalized for other types of models. We do this by asking a series of questions which mirror the questions a researcher may ask themselves when confronted with estimating an MSM of a new system, these are: 

\begin{itemize}
    \item How do MSM timescales vary with the hyperparameters? 
    \item How sensitive are timescales to the hyperparameters? 
    \item Is Bayesian optimisation a useful tool to optimise hyperparameters of an MSM? 
    \item Are the current variational scores (VAMP, GMRQ) appropriate for optimising hyperparameters? 
    \item How does model selection depend on choices in the variational score? 
\end{itemize}

We build on previous work in this area~\cite{Optimized_2016} and use as our motivating example the fast folding protein BBA from the benchmark data set provided by D. E. Sharw~\cite{lindorff-larsen_how_2011}. 
The remainder of this work is structured as follows.  In section 2 we cover the necessary theory to understand MSMs and Bayesian optimisation of hyperparameters; section 3 describes the methods; section 4 discusses the results and section 5 concludes with some recommendations. 

\section{Theory}\label{theory}
\subsection{Markov state models}
\subsubsection{Overview of MSMs}

What follows is a brief overview of the theory of MSMs, for a more detailed picture see some of the many good references~\cite{prinz_believe_2011, trendelkamp-schroer_estimation_2015}. Markov state models are describe the first order conformational kinetics of a system by specifying the conditional probability of transitioning from a state $i$ at a time $t$ to a state $j$ at a time $t+\tau$  later. This information is summarized in the transition matrix $T_{i, j}(\tau) = P(j, t+\tau | i, t)$. Each state, $i$, are collections of conformations centered around a point in a relevant feature space, e.g., conformations with similar values of backbone dihedral angles. The transition matrix is a finite and discrete representation of the underlying Markovian transfer operator, $\mathcal{T}(\tau)$, which describes the dynamics of the system. The first left eigenvector $\phi_1$ (in descending eigenvalue order, with $\lambda_{1} = 1$) corresponds to the stationary or equilibrium distribution, which we also label $\pi$; the second left eigenvectors, $\phi_2$ corresponds to the slowest conformational relaxation process (e.g., protein folding); the third is the next slowest relaxation process and so on. The eigenvalues are related to the timescales of these relaxation processes by: $ts_{i} = -\tau/\log{\lambda_i}$.  The transition matrix is said to be reversible if it obeys detailed balance $\pi_i T_{i, j}=\pi_j T_{j, i}$. 

The transition matrix is specified with respect to a set of $p$ basis states, $\chi_1, \chi_2, ..., \chi_p$ which we denote as a vector $\bm{\chi}$. In what follows, the basis states are assumed to be discrete and orthonormal and each one corresponds to a small region of conformational space (although this is not necessary).  Each frame of an MD trajectory can be mapped to one of these basis states and these discretized MD trajectories form the data from which the transition matrix is estimated.

The mapping between the atomic coordinates $\mathbf{x}$ and the basis states we call $f(\mathbf{x}; \bm{\theta}) =  \bm{\chi}$ where $\bm{\theta}$ is a vector of parameters of that mapping. For example, $f$ may involve projecting coordinates onto the backbone dihedral angles of a protein, following by clustering into \num{100} discrete states using k-means clustering. The MSM is then specified with a lag time of \SI{10}{\nano\second}. The parameters of the MSM are the \num{10000} elements of $\mathbf{T}$, while the hyperparameters are $\bm{\theta}=(\mathrm{backbone-dihedrals}, \mathrm{k-means}, 100)$ where the elements correspond to the feature, clustering method, number of basis states respectively.  

\subsubsection{Estimating a reversible MSM}

The first step in estimating a reversible MSM is projecting the MD trajectories onto the proposed basis states, $\bm{\chi}$. Transitions between each basis states at time $t$ and time $t + \tau$ are tabulated in a count matrix, $\mathbf{C}_{0t}$ (the subscript $0$ and $t$ refer to the fact that the counted transitions are between $t$ and $t+\tau$). The population of each state is given by the diagonal matrix, $\mathbf{C}_{00}$ calculated as the row-sum of the count matrix $[\mathbf{C}_{00}]_{i, i} = \sum_j [\mathbf{C}_{0t}]_{i, j}$.  A \emph{non-reversible} transition matrix is then given by $\mathbf{T}^{\mathrm{irrev}} = \mathbf{C}_{0t}\mathbf{C}_{00}^{-1}$. It is non-reversible because of the finite amount of simulation data will not be in perfect equilibrium. A transition matrix and stationary vector which obey detailed balance, $\mathbf{T}^{\mathrm{rev}}$ and $\bm{\pi}^{\mathrm{rev}}$, can be estimated from $\mathbf{C}_{0t}$ using maximum likelihood estimation with constraints~\cite{trendelkamp-schroer_estimation_2015}. The constraints ensure that detailed balance is obeyed by $\mathbf{T}$ and its dynamics are reversible.  However, once $\mathbf{T}^{\mathrm{rev}}$ and $\bm{\pi}^{\mathrm{rev}}$ have been estimated, they are now inconsistent with $\mathbf{C}_{0t}$ and $\mathbf{C}_{00}$. 

\subsubsection{Variational scores}

The key idea behind variational scores is that  approximations to the true eigenvectors of the transition matrix will given rise to eigenvalues which are bounded from above by the true eigenvalues, specifically~\cite{mcgibbonVariationalCrossvalidationSlow2015, wuVariationalApproachLearning2020c}: 
\begin{equation}\label{eqn:var_principle}
    \sum_{i=1}^{k}\hat{\lambda}_{i}^{r} \leq \sum_{i=1}^{k}\lambda_{i}^{r}
\end{equation}
where $\hat{\lambda}$ are the eigenvalues estimated from an approximate basis set $\bm{\chi}$ and $\lambda$ are the true eigenvalues. The sum runs over the first $k$ eigenvalues, which are typically the dominant slow relaxation processes that one is interested in approximating; while $r$ is some arbitrary positive integer\cite{wuVariationalApproachLearning2020c}.

When $r=1$ and the model is assumed to be stationary\cite{mcgibbonVariationalCrossvalidationSlow2015}, the left-hand side of equation~\ref{eqn:var_principle} is known as the Generalized Matrix Rayleigh Quotient (GMRQ):

\begin{equation}
    \operatorname{GMRQ}(\bm{\theta}) = \operatorname{Tr}\left[(\mathbf{U}^{T}\mathbf{C}_{01}\mathbf{U})(\mathbf{U}^{T}\mathbf{C}_{00}^\mathbf{U})^{-1}\right], \label{eqn:gmrq_def}
\end{equation}

where $\mathbf{U}$ is the matrix of eigenvectors of $\mathbf{T}$. The functional dependence of the GMRQ on $\bm{\theta}$ is to emphasize that the eigenvectors and count matrices are dependent on the hyperparameters. 

The variational approach to Markov processes placed reversible and stationary MSMs in a broader context of Koopman models which may or may not be reversible or stationary.  In this context there is a family of variational scores, differentiated by a positive integer $r$: 
\begin{equation}
     \operatorname{VAMP-r}(\bm{\theta}) = \left \| (\mathbf{U}^{T}\mathbf{C}_{00}\mathbf{U})^{-\frac{1}{2}}(\mathbf{U}^{T}\mathbf{C}_{0t}\mathbf{V})(\mathbf{V}^{T}\mathbf{C}_{tt}\mathbf{V})^{-\frac{1}{2}} \right \|_{r}^{r}, \label{eqn:vamp_def}
\end{equation}

where $\mathbf{C}_{tt}$ is the column-sum of the count matrix $[\mathbf{C}_{tt}]_{i, i} = \sum_i [\mathbf{C}_{0t}]_{i, j}$; $\mathbf{U}$ and $\mathbf{V}$ are the left and right singular vectors of the transition matrix. The matrix norm denotes takes the $r$'th power of the Schatten-r norm: 
where
\begin{equation}
    \left | T \right |_{r}^{r} = \sum_{i}s_i^r(T)
\end{equation}
where $s_i$ are the singular values of a matrix, $T$.  

If the data are stationary, reversible and $r=1$ this is equivalent to the GMRQ. With $r=2$ this expression measures the kinetic variance~\cite{noeKineticDistanceKinetic2015} captured by the basis sets. The VAMP-r scores have also been adapted to score the models based on the type of feature alone (rather than scoring the full MSM)~\cite{scherer_variational_2019}. 

As the timescales are monotonic functions of the eigenvalues, maximizing the sum of the timescales also maximizes the VAMP scores. 

\subsubsection{Cross-validation and bootstrapping}

Hyperparameters should be chosen to maximize the performance of a model on unseen data. Simply maximizing the variational score on the data used to fit the model (training data) may result in eigenvectors which describe this data well but do not generalize to new data generated by the same system. This is known as over-fitting and is a well documented phenomenon\cite{friedman2001elements}. To overcome this problem the estimated VAMP scores should close to those which would be attained on unseen data. One estimation method is to withhold a portion of the data (test set) and calculate the variational scores on this set. While accurate, it requires us ignore a large proportion of the data for training purposes, which may be wasteful when there are only a handful of observed transitions which we are interested in modelling. 

Two other popular methods, which make more efficient use of the available data,  are cross-validation~\cite{arlotSurveyCrossvalidationProcedures2009} and bootstrapping~\cite{efronIntroductionBootstrap1993}. The estimators for the variational scores (equations~\ref{eqn:gmrq_def} and \ref{eqn:vamp_def}) were both adapted to be used with cross-validation\cite{wuVariationalApproachLearning2020c, mcgibbonVariationalCrossvalidationSlow2015}: data is randomly split into two equally sized subsets. The eigenvectors vectors $\mathbf{U}/\mathbf{V}$ are calculated on one set, while the count matrices $\mathbf{C}_{00/0t/tt}$ are calculated on the other set.  This is repeated a $N_c$ times (e.g., $N_c =50$~\cite{scherer_variational_2019}) and an average of the VAMP scores taken.

The bootstrap does not require a reformulation of estimators. Instead, a number, $N_b$, of new datasets are created from the original dataset ($N_b = 100 - 1000$~\cite{efronIntroductionBootstrap1993}) and the mean of variational scores on each of these datasets used. To create the bootstrapped datasets, 
trajectories are split into small independent sub-trajectories. The sub-trajectories are sampled, \emph{with replacement} to create a new bootstrapped dataset of the same size as the original. 

Both techniques are used in practice, however, the approach we  advocate is the bootstrap, primarily because the data splitting used in cross-validation means only \SI{50}{\percent} of the data is used in estimation.  This decreases the precision of the estimated eigenvectors, and, for particularly rare events increases the likelihood  that conformational transitions may only occur in on portion of the data at a time. The sub-spaces spanned by the $\mathbf{U}/\mathbf{V}$ and $\mathbf{C}_{00/0t/tt}$ may be different, invalidating the score. In addition, the bootstrap can be used for a wider variety of observables 
 
\subsection{Hyperparameter optimisation}

\subsubsection{Methods for optimizing hyperparameters}

Finding the best set of hyperparameters $\bm{\theta}$ using either the VAMP scores or implied timescales (we will use the term \emph{response} generally), is a black-box optimisation problem.  It is black-box because we do not (in general) have access to the gradients, $\nabla_{\bm{\theta}} \operatorname{VAMP-r}$, which would facilitate a gradient based optimisation.  There are three broad classes of optimisation techniques in this case: exhaustive searching,  model based searching and population based algorithms. 

Examples of exhaustive searching grid search (popular with MSMs[all the refs]) where hyperparameters are taken from a uniformly placed grid over the hyperparameter search space, and random search, where hyperparameters are randomly sampled from the search space. 

Grid search is an effective strategy when the response is sensitive to all the hyperparameters.  However, it has poor scaling with the number of hyperparameters ($N^d$, where $N$ is the number of grid points per hyperparameter and $d$ is number of hyperparameters), so when only a small subset of hyperparameters are relevant, random search is more efficient~\cite{bergstra_jamesbergstra_random_2012}.  

A popular type of model based search is Bayesian optimisation. The Bayesian optimisation algorithm is applied to MSM estimation is as follows: 

\begin{enumerate}
    \item Randomly sample a small set of hyperparameters and measure the response of the resulting MSMs. This gives a hyperparameter trial data-set $\mathcal{D}_{n}=\left\{(y_1, \bm{\theta}_1),  \ldots (y_n, \bm{\theta}_n) \right \}$ where $y$ is the model response.
    \item Fit a regression model called a \emph{response surface}, which predicts $y$ as a function of $\bm{\theta}$ using the data $\mathcal{D}$: $y \simeq \hat{S}(\theta) + \epsilon$ (where $\epsilon$ is some error term). 
    \item \label{step:calc_alpha} Calculate an acquisition function, $\alpha$, which is a function of the response surface: $\alpha=\alpha\left[\hat{\bm{\theta}}\right]$. The acquisition function maps the hyperparameters to their utility towards optimising $y$. In other words, it suggests hyperparameters that are likely to optimize $y$. 
    \item Use $\alpha$ to suggest a set of hyperparameters, $\bm{\theta}_{n+1} = \argmax_{\bm{\theta}}{\left[\alpha(\hat{S})\right]}$ and measure the response, $y_{n+1}$ by fitting the MSM.  
    \item \label{step:reestimate_rs} Re-estimate the response surface with the new observations incorporated into the trial data-set, $\mathcal{D}_{n+1}$
    \item Repeat steps~\ref{step:calc_alpha} to \ref{step:reestimate_rs} until convergence is reached in the variational score. 
\end{enumerate}

Gaussian process regression models  are popular as response surface models and these will be discussed below. There are many different acquisition functions, in this work we use the expected improvement.  The improvement, $I$, is the one-sided difference between the current best trial value, called the incumbent, $y^{*} = \max_{\bm{\theta}\in \mathcal{D}}{S(\bm{\theta})}$ and a value of the score: $I(\bm{\theta}) = \max{\left(\hat{S}(\bm{\theta}   -y^{*}, 0\right)}$. The expected improvement is the expectation of this value after integrating out the uncertainty in the response surface $\hat{S}$. It takes into account both the size of the improvement and its probability of occurring.  The third class of optimisation algorithms are population algorithms, which include evolutionary algorithms [], particle swarm optimisation [] and covariance matrix adaption [], these will not be explored here further. 

\subsubsection{Gaussian process regression}

Gaussian process regression (GPR) models an outcome, $y$, (in our case the variational score) as a function of inputs, $\bm{\theta}$, (in our case a vector representing the MSM hyperparameters) in the form of a multivariate normal distribution: 
\begin{equation}
   y = \hat{S}(\bm{\theta}) \sim \mathcal{N}\left(\mu(\bm{\theta}), \mathbf{K}\right ) \label{eqn:gp_def}
\end{equation}
where $\mu(\bm{\theta})$ is the mean function and $\mathbf{K}$ matrix which specifies the covariance between different, arbitrary, values of the outcome, $y$ at different values of $\bm{\theta}$. To understand this expression, first specifying a set of points of $\bm{\theta}$ and a value of $\mu$, e.g., $\mu=0$. Then sampling from the resulting multivariate normal distribution gives rise to a mapping between $\bm{\theta}$ and $y$.

However, as written, equation~\ref{eqn:gp_def}, contains no information.  To make predictions about new points, $(y_{*}, \bm{\theta}_{*})$ (the asterisk denotes unseen data), training data $\mathcal{D}_{n}=\left\{(y_1, \bm{\theta}_1),  \ldots (y_n, \bm{\theta}_n) \right \}$,  needs to be incorporated into the definition of  $\mu$ and $\mathbf{K}$:  

$$
\begin{aligned}
\mu & =K\left(\bm{\theta}_*, \bm{\theta}\right)\left[K(\bm{\theta}, \bm{\theta})+\sigma_n^2 I\right]^{-1} \mathbf{y} \label{eqn:gpr_pred_mu} \\
\mathbf{K} &=K\left(\bm{\theta}_*, \bm{\theta}_*\right)-K\left(\bm{\theta}_*, \bm{\theta}\right)\left[K(\bm{\theta}, \bm{\theta})\right]^{-1} K\left(\bm{\theta}, \bm{\theta}_*\right), \label{eqn:gpr_pred_cov}
\end{aligned}
$$
here $K\left(\bm{\theta}_*, \bm{\theta}\right)$ is the covariance between the response at some arbitrary new points, $\bm{\theta}_*$, and the points in training data $\bm{\theta}$ (and similarly $K\left(\bm{\theta}_*, \bm{\theta}_*\right)$, and $K\left(\bm{\theta}, \bm{\theta}\right)$). The GPR model thus specified predicts a response $y_*$, given a new set of inputs, $\bm{\theta}_*$, as a Gaussian distribution, with a mean and variance derived from equations~\ref{eqn:gpr_pred_mu} and~\ref{eqn:gpr_pred_cov}. 

In order to calculate the covariance between the response in the training data and the response of new points, a covariance kernel is used. As an example, equation~\ref{eqn:gauss_kernel} shows the Gaussian kernel: 
\begin{equation}
    k(\theta_i, \theta_j; l) =  \exp\left(-\frac{\left|\theta_i-\theta_j\right|^2}{l^2}\right), \label{eqn:gauss_kernel}
\end{equation}
it calculates the covariance between the response $y_i$ and $y_j$ with input values of $\theta_i$ and $\theta_j$. It is parameterized by a length-scale parameter, $l$, which is learned from the training data. To increase the flexibility of the covariance calculation, equation~\ref{eqn:gauss_kernel} can be augmented to: 
\begin{align}
    K(\theta_i, \theta_j) & = \eta^2  k(\theta_i, \theta_j; l) + \sigma^2\delta_{ij}, 
\end{align}
here $k(\theta_i, \theta_j)$ is a covariance kernel (e.g., a Gaussian kernel),  $\theta_i$ and $\theta_j$ are two values of the inputs, $\eta$ determines the scale of the correlation in response, $l$ is the characteristic lengths scale of the Gaussian process, and $\sigma$ is the variance of a term which models the noise in the \emph{observed} response. 

There are many different types of kernel with different functional forms depending on the type of data (e.g., periodic data). When the covariance depends only on distance between observations (as above) the kernel is known as \emph{stationary}.  We will only consider stationary kernels in this work and these will be discussed in the methods section. 

The parameters of the GPR ($\eta, l, \sigma$ in the above case) are learned through maximizing the marginal log-likelihood of the data with respect to the parameters. Typically, prior distributions are placed over these parameters which reflect prior knowledge or restrictions on the data. Details on the fitting process can be found in [rasmussen and williams]. 

With multidimensional inputs (i.e., multiple MSM hyperparameters) then there is flexibility over how to create a kernel over all the input dimensions. Two simple approaches is a fully additive ($K^{\mathrm{+}}$) and a fully multiplicative ($K^{\mathrm{\times}}$) covariance function: 

\begin{align}
    K^{\mathrm{+}}_{i,j} & = \eta^2  \left [k(\theta^{1}_i, \theta^{1}_j; l_{1}) + k(\theta^{2}_i, \theta^{2}_j; l_{2}) \ldots \right ] + \sigma^2  \label{eqn:plus_kernel} \\ 
    K^{\mathrm{\times}}_{i,j} & = \eta^2  \left [k(\theta^{1}_i, \theta^{1}_j; l_{1}) \times k(\theta^{2}_i, \theta^{2}_j, l_{2}) \times \ldots \right ] + \sigma^2 \label{eqn:mult_kernel}
\end{align}

where the elements of $\bm{\theta}$ are labelled $\theta^{1}, \theta^{2} \ldots$. The different kernel constructions lead to different interpretations of covariance structures.  Kernel functions can by combined in arbitrary ways to suite the modelling needs, see [duvenand thesis]. 

\subsubsection{Sensitivity analysis}

Sensitivity analysis is used to determine the dependency of model outputs on their inputs [uncertainty book]. This is a large research area and we refer the reader to [uncertainty book] and the references within for a full account.  In this work, we conduct a \emph{global sensitivity analysis} which looks at model outputs as a function of the whole input space. This is in contrast to a \emph{local sensitivity analysis} which look at  how small perturbations around a give set of inputs affect the outcomes.  Our simplified sensitivity analysis is based on the work of [bergstra] which modelled the accuracy of a deep neural network image classifier in response to changes in its hyperparameters (e.g., the learning rate) as Gaussian process.  

The sensitivity of the hyperparameters are calculated from the learned parameters of the covariance matrix, specifically the values of $l$ for each term in equations~\ref{eqn:plus_kernel} or \ref{eqn:mult_kernel}.  $l_{i}$  determines how correlated the response is as a function of the distance between two different points of the $i$'th hyperparameters $\theta^{i}$. For a large value of $l_{1}$, two observations $y_{i}, y_{j}$ will be, on average, correlated with one another for a given large values of $|\theta_{i}^{1} - \theta_{j}^{1}|$. This means $y$ does not vary significantly with changes in $\theta^{1}$. The converse will also be true: for small $l_{1}$, $y$ will change significantly for given large value of $|\theta_{i}^{1} - \theta_{j}^{1}|$. Thus, the sensitivity of the response to hyperparameter $i$ can be measured by the \emph{relevance}, $R_{i}$: 
\begin{equation}\label{eqn:relevance_def}
    R_{i} = \frac{1}{l_{i}}
\end{equation}

\section{Methods}
\subsection{Molecular dynamics}

This work uses simulation data of the fast folding protein, BBA, one of the twelve fast-folding proteins which have become the de-facto benchmark dataset for testing molecular kinetics methods. The methods used to create this data are described elsewhere~\cite{lindorff-larsen_how_2011}. Important information on the data are shown in table~\ref{tab:data_description}. The average folding time was calculated by the authors~\cite{lindorff-larsen_how_2011}; the sub-trajectory length and number of sub-trajectories correspond to the data splitting used in bootstrapping calculation of the uncertainty in model observables.

\begin{table}
    \caption{\textsc{Description of molecular dynamics data}}
    \begin{tabularx}{\textwidth}{llXXXXX}
    \toprule
    Name & PDB & Simulation time (\si{\micro\second}) & Average folding time (\si{\micro\second}) & No. Residues & Sub-trajectory length (\si{\micro\second}) & No. sub-trajectories \\
    \midrule
    BBA                 & 1FME      & \num{325}     & \num{18}  & 28 & \num{2} & 164 \\
    \bottomrule
    \end{tabularx}
    \label{tab:data_description}
\end{table}

\subsection{Markov state models}
MSMs were estimated using PyEMMA version 2.5.7~\cite{schererPyEMMASoftwarePackage2015a} an using a standard pipeline when focusing on the slow relaxation processes~\cite{noe_markov_2019, husic_markov_2018}: 
\begin{enumerate}
    \item Project molecular dynamics (MD) trajectories onto a set of features. 
    \item Reduce the dimension of the feature trajectories using TICA with a lag time $\tau_{\mathrm{TICA}}$ by projecting onto the first $m$ TICA coordinates. 
    \item The frames of the TICA trajectories were clustered using the k-means algorithm into $n$ discrete microstates. 
    \item A reversible, maximum likelihood MSM was then estimated. 
\end{enumerate}
To save on memory and compute resources  the data was subset in parts of the MSM estimation. The MD trajectories were first strided so that each frame corresponded to \SI{1}{\nano\second}. The cluster centers were estimated on frames separated by \SI{10}{\nano\second}, i.e. only the 0th, 10th, etc. frames were used for estimating the cluster centers. 

The uncertainty for model observables (e.g., implied timescales, VAMP2 scores etc.) was estimated using the bootstrap with \num{100} bootstrap samples. The point estimate and error-bars  were calculated as the median,   \SI{2.5}{\percent} \& \SI{97.5}{\percent} quantiles of the distribution over the bootstrap samples.

\subsection{Hyperparameters and scoring}
\num{140} different hyperparameters were randomly sampled from the search space described by table~\ref{tab:search_space}. Each set of hyperparameters and their corresponding model observables are known as a \emph{hyperparameter trial} (or just \emph{trial}). Three different features, $f$, were used: dihedrals feature (`dihed.'), contact distance feature (`dist.') and a logitistic transformation of the contact distances (`logit(dist.)').  The number of trials for each feature is dependent on the dimension of the hyperparameters space which will be explained below. 

The dihedral angles feature used all the available backbone and residue dihedral angles (except the $\omega$ angle). The contact distance (used in the `dist.' and `logit(dist.)' features) is denoted, $d$; the logistic transform $\mathrm{logit}(d) = [1-\exp{(s(d-c))}]^{-1}$, where center, $c$, and steepness $s$,  have units of \si{\angstrom} and \si{\per\angstrom} respectively. The definitions of the contact distances ($d$) were either the closest heavy-atom distance ($X-X$) or the distance between the $\alpha$-Carbons (C$\alpha$-C$\alpha$). The TICA eigenvectors were scaled by their eigenvalues ($\lambda$) so that distances in TICA space correspond to kinetic distances~\cite{noeKineticDistanceKinetic2015}.

The number of trials was approximately proportional to the number of hyperparameters for each feature: 20 trails for the dihedral feature, 40 for the contact distances (20 for each value of the contact distance scheme: $X-X$,  C$\alpha$-C$\alpha$) and 80 for the logistic transformation of contact distances (which, in addition to the two distance scheme values, has two other hyperparameters, $c$ and $s$). 

For each trial,  $\bm{\theta} = (f, \tau_{\mathrm{T}}, m, n, c, s)$,  an MSM was estimated using the procedure above with a range of Markov lag-times, $\tau_{\mathrm{M}}$: \SI{1}{\nano\second}, \SI{11}{\nano\second}, ..., \SI{101}{\nano\second}. For each combination of $\bm{\theta}$ and  $\tau_{\mathrm{M}}$ the slowest \numrange{2}{21} eigenvectors were scored using the VAMP2(k) (equation~\ref{eqn:vamp_def}) and  VAMP2$_{eq}$(k) score (equation~\ref{eqn:vamp_eq_def}):
\begin{equation}
    \operatorname{VAMP2_{eq}}(k) = \sum_{i=1}^{k}\lambda_{i}^{2}, \label{eqn:vamp_eq_def}
\end{equation}
along with the implied timescales, $t_i$.  Each of these observations were estimated as the median of $N_b=100$ bootstrapped samples. 

\begin{table}
    \centering
    \begin{tabularx}{\textwidth}{lXXXX}
    \toprule
    \textbf{Features}, ($f$)  & & & &\\
    Dihedral angles & \textsc{Which} & & &\\
    & \multicolumn{2}{l}{$dihed.=\phi, \psi, \chi_{1}, \ldots, \chi_{5}$ } & & \\
    Contact distances &  \textsc{Definition}, ($d$) & \textsc{Transform}& \textsc{Center} ($c$, \si{\angstrom}) & \textsc{Steepness} ($s$, \si{\per\angstrom}) \\

     & \nextitem $X$-$X$  \nextitem C$\alpha$-C$\alpha$ & \nextitem $\mathrm{logit}(dist.)$ \nextitem $dist.$ &  \numrange{3}{15} & \numrange{0.01}{5} \\
    \midrule
    \textbf{Decomposition} & \textsc{Eigenvectors}, ($m$) & \textsc{Lag-time}, ($\tau_{T}, \si{\nano\second}$) & \textsc{Scaling}\\ 
    TICA & \numrange{1}{20} & \numrange{1}{100} & $\lambda$\\
    \midrule
    \textbf{Clustering} & \textsc{Clusters}, ($n$) &\\
    k-means & \numrange{10}{1000} & \\
    \bottomrule
    \end{tabularx}
    \caption{\textsc{Hyperparameter search space}. $X$-$X$ and C$\alpha$-C$\alpha$  refer to the closest heavy atom and $\alpha$-Carbon scheme respectively, for measuring the contact distance ($dist.$).  }
    \label{tab:search_space}
\end{table}

The hyperparameter trial data-set, $\mathcal{D}$, consisted of: $100$ bootstrap samples of $140$  unique sets of hyperparameters, at $10$ different lag times, with  $20$ measurements of the the implied timescales and $20$ measurements of the VAMP2 score and $20$ measurements of the VAMP2$_{eq}$ score. The total number of these observations ($t_i$, VAMP2(k), VAMP2$_eq$(k)) is therefore: $100 \times 140 \times 10 \times (20 + 20 + 20) = \num{8400000}$. 

\subsection{Markov lag time}
The Markov lag time, $\tau_{\mathrm{M}}$, was calculated from the total hyperparameter trial data-set. For each trial the following gradient was calculated:
\begin{equation}
    g(\tau_{\mathrm{M}}, \theta) = \frac{\Delta \log{\left(t_{2}(\tau_{\mathrm{M}}, \theta)\right)}}{\Delta \tau_{\mathrm{M}}}, 
\end{equation}\label{eqn:choose_lag_1}
The selected Markov lag-time, $\tau^{*}_{\mathrm{M}}$ was chosen as:
\begin{equation}
    \tau^{*}_{\mathrm{M}}  = \argmin_{\tau_{\mathrm{M}}, \theta}\left[g(\tau_{\mathrm{M}}, \theta)\right], \quad 0 < g < \log{1.01}
\end{equation}\label{eqn:choose_lag_2}
This codifies and extends the generally accepted process by which the implied timescales $t_{i}$ as a function of $\tau_{\mathrm{M}}$ are plotted on a log scale and the smallest $\tau_{\mathrm{M}}$ for which $t_{2}$ is constant is chosen. Our extension is that we consider a range of different values of $\theta$. 

\subsection{Sensitivity analysis}

The hyperparameter trial data-set, $\mathcal{D}$, was used to estimate the sensitivity of the dominant timescale, $t_2$, to the quasi-continuous hyperparameters (i.e., everything except the type of contact distance scheme, although this was included in the GPR modelling), for each feature, $f$, separately. For example, the relevance (equation~\ref{eqn:relevance_def}) of the TICA lag-time, $\tau_{\mathrm{T}}$, in determining $t_2$ was calculated for the `dihed.', `dist.' and `logit(dist.)' feature.  The relevance of each hyperprameter was calculated from the characteristic length-scales in a GPR fitted to $\mathcal{D}$.  

A number of different covariance structures were trialled when fitting the GPRs to $\mathcal{D}$. Four different types of kernels over each hyperparameter were trialled (see equations~\ref{eqn:kern_exp} to~\ref{eqn:kern_gauss}, below) and combined in both a fully multiplicative and fully additive covariance matrix (equations~\ref{eqn:mult_kernel} and~\ref{eqn:plus_kernel} respectively).  So for each feature, eight different GPR models were fit. The final form of the covariance matrix elements were chosen by looking at the RMSE of the predictions across all features. The three different kernel functions were taken from the Mat\'ern family with $\nu=\sfrac{1}{2}, \sfrac{3}{2}, \sfrac{5}{2}, \infty$ ($\nu=\sfrac{1}{2}$, and $\nu=\infty$ correspond to the Exponential and Gaussian kernels respectively), these are given by: 
\begin{align}
k_{\text{Exp}}\left(r; \sfrac{1}{2}\right) &=\exp (-r) \label{eqn:kern_exp}\\
k_{\text{M3-2}}\left(r; \sfrac{3}{2}\right) &= \exp (-\sqrt{3} r)(1+\sqrt{3} r) \label{eqn:kern_m32} \\
k_{\text{M5-2}}\left(r; \sfrac{5}{2}\right) &= \exp (-\sqrt{5} r)\left(1+\sqrt{5} r+\frac{5}{3} r^{2}\right) \label{eqn:kern_m52}\\
k_{\text{RBF}}\left(r; \infty\right) &= \exp \left(-\frac{1}{2} r^{2}\right), \label{eqn:kern_gauss}
\end{align}
where $r = \frac{|\theta_i-\theta_j|}{l}$. See chapter 5 of  reference~\cite{rasmussenGaussianProcessesMachine2006} for a full description of the Mat\'{e}rn kernels and their properties.  

The hyperparameter trial data-set was processed with the following steps prior to modelling (this was done using the data for all features):
\begin{enumerate}
    \item The median of $t_2$ across the bootstrap samples was taken as the response variable. 
    \item The response  was log-transformed and then centred and scaled by its median and inter-quartile range, respectively. This was because of its large range: $t_2$ ($10 < t_2 < \num{20000}$. 
    \item The input variables ($m$, $\tau_{T}$, \ldots, etc.) were scaled to have the same range as the transformed response. This was done so that the kernel hyperparameters could be comparable to each other. The exception to this was contact distance scheme which was left dummy coded as $1$ (indicates trial used closest-heavy distance) and $0$ (trial used alpha-Carbon distance). 
\end{enumerate}

The GPR was fit using PyMC v4.0 [need ref] using the marginal likelihood implementation.  The values of the GPR parameters were found by maximizing the marginal likelihood using the `find$\_$MAP()' function. An informative prior was placed over the kernel parameters: a half-Cauchy distribution with a scale parameter of $1$ for $\sigma$ and $\eta$, and a Gamma distribution with $\alpha, \beta$ = $1, 0.5$. The informative priors and the scaling of the response and hyperparameter inputs were done to to ensure the MAP algorithm consistently converged to an answer with a low RMSE. Without these measures the `find$\_$MAP()' frequently predicted the response to be zero or near zero. 

After fitting a GPR model and selecting the best fitting covariance matrix form, the uncertainty in the covariance parameters $\sigma, \eta, l_{1}, l_{2}\ldots$ were determined by bootstrapping with $100$ iterations. 

 
\subsection{Bayesian optimisation}
 
Bayesian optimisation was applied to optimize the hyperparameters conditional on a given feature. This was done as follows: 

\begin{enumerate}
    \item The GPR model used in the sensitivity analysis was used to compute the expected improvement \emph{at the values of $\theta$ in $\mathcal{D}$} according to the following: 
    \begin{align}
     \mathbb{E}[I(\bm{\theta})] &= \mathbb{E}[\max{(0, f(\bm{\theta})-f^{*})}] \label{eqn:ei_def} \\ 
     & = \sigma \left ( z \Phi(z)  + \phi(z) \right) \label{eqn:ei_for_gp}
    \end{align}
    where $z = (\mu-\mu^{*})/\sigma$ and $\mu$, $\sigma$ are the mean and standard deviation of the GPR at a given point, and $\mu^{*}$ is the incumbent. 
    \item The maximum of the expected improvement over whole of the hyperparameter search space would be too computational inefficient to calculate. So we assumed that the maximum would be in the neighbourhood of the incumbent set of hyperparameters (i.e., the maximum $\mathbb{E}[I(\bm{\theta})]$ restricted to those $\bm{\theta}\in \mathcal{D}$). 
    \item The neighbourhood of the incumbent was determined as the range of $\bm{\theta}$ which have an $\mathbb{E}[I(\bm{\theta}])]$ within \SI{5}{\percent} of the incumbent. 
    \item A grid of new hyperparameters was place over this neighborhood and the expected improvement was calculated at each grid point.  This gave a list of new hyperparameters which were ordered according to their expected improvement.  MSMs using the top 5 unique hyperparameters were then calculated. 
\end{enumerate}

The code used to create the hyperparameter trial dataset, $\mathcal{D}$ can be found at \url{https://github.com/RobertArbon/msm_sensitivity} and the code used to perform all other anlaysis can be found at \url{https://github.com/RobertArbon/msm_sensitivity_analysis}.  

\section{Results and discussion}

We start from a position of limited information on appropriate modelling choices for creating an MSM of the fast folding protein BBA.  We assume that the
relevant hyperparameters are contained in the  hyperparameter search space, table~\ref{tab:search_space} and set out to answer our questions listed in the introduction.  Our motivation in answering these questions is ensure that an MSM analysis (or indeed any statistical analysis of MD data) is robust and reproducible.  

Unless otherwise stated, the Markov lag time, $\tau_{\mathrm{M}}$ was set at \SI{41}{\nano\second} which was the smallest lag time across the whole hyperparameter trial data-set which gave rise to a Markovian model.  

\subsection{How do MSM timescales vary with hyperparameters?}

\begin{figure}
    \centering
    \includegraphics[width=0.8\textwidth]{figures/1fme_timescales.pdf}
    \caption{Slowest timescale $t_{2}$ extracted from each trial MSM. Panel (a) shows the randomly sampled hyperparameter trials. Panel (b) includes the results of one iteration of Bayesian optimisation (original trials shown as transparent). Panel (c) includes the results of one iteration of Bayesian optimisation excluding the influence of the best ranked trial in panel (a).  Central value and error bars are derived from 100 bootstrap samples as the median, the \SI{95}{\percent} quantiles. The blacked dashed line is the value  reported in~\cite{lindorff-larsen_how_2011} and is derived directly from the MD trajectories.} 
    \label{fig:1fme_timescales}
\end{figure}

We would like to know how much variation in timescales exists across the hyperparameter search space in a model observable. Figure \ref{fig:1fme_timescales}, panel (a), show the slowest timescale, $t_2$, extracted from each of the MSMs. The three features are denoted by colour and the trials are ordered according to the median $t_2$. For reference, the average folding timescale taken from a an analysis of the MD trajectories~\cite{lindorff-larsen_how_2011} is \SI{18}{\micro\second} and is shown as a dashed line.  

A number of features of the distribution are clearly apparent: First, there are almost three orders of magnitude variation in $t_2$ in `logit(dist.)' feature. Second, only a single trial  a timescale close to this value (\SI{20}{\micro\second}, unsigned error: \SI{14}{\percent}), while the other trials an unsigned error of  at least \SI{45}{\percent}. The trials using the `logit(dist.)' feature make up the majority of both the best and worst performing hyperparameters; while the dihedral and contact distances sit within the middle of the distribution of timescales (see also figure~\sref{fig:ts_distribution}). This suggests that only a small volume of the hyperparameter search space gives rise good hyperparameters.  

These observations indicate that $t_2$ is particularly sensitive to at least one hyperparameter with the `logit(dist.)' feature and that there is room for optimising the hyperparameters still further. 

\subsection{How sensitive are timescales to the MSM hyperparameters?}\label{sec:sensitivity}

\begin{figure}
    \centering
    \includegraphics[width=0.8\textwidth]{figures/sensitivity.pdf}
    \caption{Hyperparameter relevance ($R$) to the timescales ($y=\log{t_{2}}$)for the three different features,  and the expected improvement ($y=\log{\mathbb{E}[I]}$) for the logistic-distances feature. The larger the relevance, the more sensitive the outcome is to each hyperparameter.  The relevance is calculated from the learned kernel parameters of the relevant Gaussian process. Box plots show the distribution of values from \num{100} bootstrap samples. The relevance $R$ is the inverse of the characteristic length-scale of the exponential kernel in equation~\ref{eqn:kern_exp}. }
    \label{fig:sensitivity}
\end{figure}

In order to understand the timescale distribution we estimate the sensitivity of the timescales to the hyperparameters. Previous work has shown that the type of feature used is important in determining the eigenvector accuracy. This is consistent with figures~\ref{fig:1fme_timescales}  where the majority of the best hyperparameters use the `logit(dist.)' feature.  However, as figure~\ref{fig:ts_distribution} also shows that a) there is significant overlap of the timescale distributions for each feature, and b) the best performing feature also gives rise to the worst performing models. The type of feature is therefore a necessary but not sufficient hyperparameter to tune in order to maximize the timescales. It is important, therefore, to consider the remaining hyperparameters (e.g., the TICA lag time, TICA dimension etc.) and how they affect the timescales.  

We estimate the sensitivity of the remaining hyperparameters for each feature by modelling $\log{t_{2}}$ as a Gaussian process as a function of the MSM hyperparameters. This model of the response of the $t_2$ to variation in the hyperparameters we call a \emph{response surface}. The covariance function that gave the best fitting model used the fully additive structure, equation~\ref{eqn:plus_kernel}, with an exponential kernel over each hyperparameters, equation~\ref{eqn:kern_exp}. 

Figure~\ref{fig:sensitivity} plots the  the relevance of hyperparameters, $R$, in determining $t_{2}$. It shows that the hyperparameters vary in how important they are in determining $t_2$ for each different feature.  For the `dihed.' feature it shows that the TICA dimension, lag time and number of microstates are all equally relevant in determining $t_2$, however $t_2$ does not vary greatly with changes in any of these hyperparameters (as $R\simeq 1$). For the `dist.' feature the number of microstates is the most relevant in determining $t_2$. For the `logit(dist.)' feature the location of the centre of the logistic function (as this feature is a continuous approximation to a contact map, the `centre' corresponds  to the contact map cut-off) is the most important hyperparameter. The fitted Gaussian process regression models which give rise these values are shown in the supplementary material figures~\ref{fig:repsonse_diheds}, \ref{fig:repsonse_dist}, and \ref{fig:repsonse_logistic}. 

Why is the hyperparameter sensitivity important? First, as has been  previously shown~\cite{bergstra_jamesbergstra_random_2012} if all the hyperparameters have a large relevance ($R \gg 1$) then to find the maximum of the response surface (i.e., the hyperparameters which give rise to the maximim $t_2$) then it is important to try all combinations of hyperparameters i.e., a  grid-search strategy, or to employ an active learning approach~\cite{snoekAbstractBayesianOptimization2013}.  However, if only a small proportion of the hyperparameters have a high relevance then randomly sampling hyperparameters is the more efficient strategy.  Second, the lack of pattern in the relevance of hyperparameters indicates that it is not possible to draw general conclusions about hyperparameters, indicating that default values of hyperparameters may not be useful and that each system must optimised. 


\subsection{Can we improve the timescale estimates?}

\begin{figure}
    \centering
    \includegraphics{figures/surface_distances_logistic_ei.pdf}
    \caption{Expected improvement as a function of the two most relevant hyperparameters - the logistic center ($c$) and steepness ($s$). The remaining hyperparameters take on the values shown in the annotation. These were chosen so as to maximize the expected improvement, i.e., the figure plots $y=f(s, c, d^{*}, m^{*}, \tau^{*}, n^{*})$ where $s^{*}, c^{*}, d^{*}, m^{*}, \tau^{*}, n^{*} = \argmax \left [f(s, c, d, m, \tau,n)\right]$. }
    \label{fig:ei_surface}
\end{figure}

In order to check the convergence of the results shown in panel (a) of figure~\ref{fig:1fme_timescales} and to potentially improve the $t_2$ estimate,  we use a single round of Bayesian optimisation. To do this, we first calculate the expected improvement, $\mathbb{E}[I]$,  using the response surface calculated in the sensitivity analysis and  equation~\ref{eqn:ei_for_gp}. As the `logit(dist.)' feature has the best performing trials we optimise only this feature. 


The expected improvement is shown in figure~\ref{fig:ei_surface} as a function of the `centre' and the `steepness' hyperparameters (the remaining hyperparameters are held at their values at the peak of expected improvement).  This shows that we expect any values of the `centre' and `steepness' hyperparameters to increase $t_2$ by at least \SI{2}{\micro\second} if the other hyperparameters take on their values in the annotation. However, with $c\simeq \SI{7}{\angstrom}$ and $s\simeq \SI{0.5}{\per\angstrom}$ we can expect an improvement in the timescales of almost $\SI{3}{\micro\second}$. 


We then estimated MSMs using the five sets of hyperparameters with the largest expected improvement. The new optimized values of $t_2$ are shown in panel (b) of figure~\ref{fig:1fme_timescales}.  These clearly show that optimized models all have timescales slightly greater than the incumbent ($t_2 \simeq \SI{20}{\micro\second}$):  the optimized timescales range up to $t_2 \simeq \SI{24}{\micro\second}$, which are similar to the improvements predicted by the expected improvement function. The optimized hyperparameters differ slightly from the incumbent (except in the choice of distance scheme - all used the $\alpha$-carbon distances). We have, therefore, a set of six MSMs which differ slightly in the values of their hyperparameters, which all give similar implied timescales: we can be sure that our model is robust. 


We can also ask ourselves whether we could improve our hyperparameters using Bayesian optimisation with less data. This is relevant if estimating the models is expensive. To answer this we remove the incumbent in panel (b) from the hyperparameter trial data-set, re-estimated the response surface and expected improvement and sampled new hyperparameters. The new incumbent in this case was $t_2 = \simeq \SI{10}{\micro\second}$, shown in purple in panel (c) figure~\ref{fig:1fme_timescales}.  The new values of $t_2$,  are all improvements of up to \SI{10}{\micro\second}: the new timescales range up to $t_2 \simeq \SI{20}{\micro\second}$. It is therefore plausible that one may use Bayesian optimisation to optimize MSM hyperparameters, even when the timescale measurements are noisy and there are no strong dependence on any of hyperparameters.  

\subsection{Are VAMP scores suitable for model selection of reversible MSMs?}
\begin{figure}
    \centering
    \includegraphics{figures/bad_vamp_ranks.pdf}
    \caption{Models with VAMP scores inversely proportional to timescales. Panels (a) and (c) show the VAMP2 scores (as calculated by equation~\ref{eqn:vamp_score_def} and as implemented in the package Deeptime~[ref]) for a selection of models where the slowest timescale is inversely proportional to the VAMP2(k=2) score. The horizontal axis is the model rank as judged by the VAMP2(k=2) score. Models which do not show this correlation are not shown.  Panels (b) and (d) show similar but for a lower ranking set of models. }
    \label{fig:bad_vamp_scores}
\end{figure}


\begin{figure}
    \centering
    \includegraphics{figures/timescale_vs_vamp_vs_evs.pdf}
    \caption{Timescales as a function of the VAMP2 scores (panel (a) and sum of square eigenvalues (b). Each point is calculated from a simulated trajectory of 20 observations using the $3\times 3$ example in Trendelkamp-Schroer et. al.~\cite{trendelkamp-schroerEstimationUncertaintyReversible2015b}. The transition matrix was estimated ensuring reversibility, all calculations were performed using the Deeptime package. There are 10000 points in total.}
    \label{fig:bad_vamps_examples}
\end{figure}


The VAMP2 score~\cite{wuVariationalApproachLearning2020c} provide a principled metric for optimising MSM hyperparameters. The benefits are that it can be used for stationary, non-stationary, reversible and non-reversible MSMs. It is linked directly to the kinetic variance captured by the basis states such that maximizing the VAMP2(k) score will maximize the timescales of pertaining to the first $k$ eigenvectors of the model. In addition it can be used with bootstrapping and cross-validation model selection techniques. 

We tested whether the VAMP scores are appropriate for model selection with reversible MSMs. For each of the hyperparameters trials we calculated the VAMP2(k=2) score so that we are focused solely on maximizing the dominant $t_2$ timescale. Inspection of the results revealed that for subsets of the trials, VAMP2(k=2) was inversely proportional to the $t_2$. This is shown in figure~\ref{fig:bad_vamp_scores}. In panel (a) the VAMP2 score is shown for the trials ranked first, third, and fourth. In panel (c) the first five timescales are shown for each model.  The dashed line shows the folding timescale from reference~\cite{lindorff-larsen_how_2011}. Timescales for the third to sixth eigenvectors are the same, however $t_2$ clearly \emph{increases} with \emph{decreasing} VAMP2 score. The second ranked model is omitted for clarity because it does not follow this pattern. 

We posit that the reason for this behaviour is due to the fact by enforcing reversibility in the estimation of the transition matrix it is impossible to get consistency between the three count matrices ($\mathbb{C}_{00, 0t, tt}$) and the eigenvectors ($\mathbb{U}/\mathbb{V}$) in equation~\ref{eqn:vamp_score_def}).  A similar effect can be seen with a three-state toy model (example 1 from~\cite{trendelkamp-schroer_estimation_2015}). \num{10000} 20-step trajectories were sampled from the same transition matrix. The implied timescales and VAMP2(k=2) scores were estimated and are shown in figure ~\ref{fig:bad_vamps_examples} panel (a).  The timescales are correlated with the VAMP2 scores but the correlation is not perfect. Many subsets of these results could form sets which are anti-correlated, in the same way as in figure~\ref{fig:bad_vamp_scores}. In panel (b) we plot the sum of the squares of the first two eigenvalues, which shows perfect rank correlation (as they must as they are analytically related to each other). 

The reason for writing the VAMP score as the product of count matrices and eigenvectors/singular vector matrices is to facilitate data-splitting in cross-validation. While we used bootstrapping for this work, the effect of data splitting would be to worsen the discrepancy between the count and transition matrices. This is because the count matrices are now estimated on different data compared to the eigenvector matrices.  

In addition to the problem of consistency between the matrices in equation~\ref{eqn:vamp_score_def} from a) enforcing reversibility and b) data splitting for cross-validation, we recommend that cross-validated VAMP scores are not used for reversible and stationary MSMs.  Instead we recommend bootstrapping the sum of the squared eigenvalues (VAMP$_{eq}$(k)) directly from the reversible transition matrix. This has the same theoretical properties of the VAMP2 score (i.e., represents captured kinetic variance, and link to variational theorem) while not wasting data due to data splitting and perfect correlation with the implied timescales. 

\subsection{Does the lag time and number of scored eigenvectors affect the model selection?}

\begin{figure}
    \centering
    \includegraphics{figures/vampeq_rank_vs_lag.pdf}
    \caption{Consistency of $\operatorname{VAMP2}_{\mathrm{eq}}(k)=\sum_{i=1}^{k}\lambda_{i}^{2}$ rank with Markov lag time, $\tau$. The $i, j$'th cell in panel (a) shows the Spearman's rank correlation coefficient of $\operatorname{VAMP2}_{\mathrm{eq}}(k=2)$ for each trial measured at the $i$'th lag time, with   $\operatorname{VAMP2}_{eq}(k=2)$  measured at the $j$'th lag time.  Only the top $50$ trials as measured at the $i$th lag time were used.  Panels (b) - (d) show the same measurements but with $k=3, 5$ and $10$  in the $\operatorname{VAMP2}_{\mathrm{eq}}(k)$ score respectively. }
    \label{fig:vamp_rank_vs_lag}
\end{figure}

When evaluating MSMs using a variational score one must specify the both the Markov lag time ($\tau$) and the number of eigenvectors to score ($k$).  However, both these choices affect the VAMP score although it is not clear  whether these choices affect the model ranking.  To test how these choices affect model selection we measured the consistency in model rank, as measured by the VAMP2$_{eq}$(k) using the Spearman's rank correlation coefficient, at a) different lag times for given values of $k$ and b) at different numbers of scored eigenvectors at a given lag time. 

Figure~\ref{fig:vamp_rank_vs_lag} shows the consistency between model rankings at different lag times ($\SI{1}{\nano\second} < \tau < \SI{101}{\nano\second}$) with $k=2$ (panel (a)) and with $k=10$ (panel (b)). In addition, scatter plots of the data used to calculate these coefficients for $k=2, 3, 5, \& 10$ are shown in the supplementary information figures~\ref{fig:vampeq2_rank_vs_lag_pairplot} to \ref{fig:vampeq10_rank_vs_lag_pairplot}.  Across all lags and for both small and large numbers of scored eigenvectors, the consistency in the model ranking is high (greater than \SI{85}{\percent}). For all but the shortest lag time ($\tau=\SI{1}{\nano\second}$) the rank correlation is much greater, ranging up to \SI{100}{\percent} for between long lag-times.  This effect is most pronounced for larger numbers of scored eigenvectors. In particular, good consistency is achieved at lag times smaller than those required for the model to be Markovian ($\tau=\SI{41}{\nano\second}$).


\begin{figure}
    \centering
    \includegraphics{figures/vampeq_rank_vs_proc.pdf}
    \caption{\textsc{Consistency of $\operatorname{VAMP2}_{eq}(k)$ rank with number of scored eigenvectors}. The ranks of trials in the row $k$ are compared to their rank at the column $k$ using the Spearman's rank correlation coefficient at a lag time of \SI{41}{\nano\second}}
    \label{fig:vampeq_rank_vs_n_procs}
\end{figure}

Figure~\ref{fig:vampeq_rank_vs_n_procs} shows the consistency between model rankings at different number of scored eigenvectors ($2 < k < 21$) at a lag time of \SI{41}{\nano\second} (the value used in all previous analysis). Again, the consistency is generally high with a rank correlation between all pairs of $k$ of at least \SI{80}{\percent}. The ranking is most consistent between values of $k$ larger than \num{4}.  From these two analyses taken together, we see that for long lag-times and large number of scored eigenvectors model ranking is not affected by the choice of $\tau$ and $k$.

  the the the         u    the is    ()thusing the 
\subsection{Conclusions}
This work has answered a number of questions pertaining to selecting hyperparameters for discrete MSMs.  We have seen that type of feature is important but that the other hyperparameters need to be chosen carefully to optimize the implied timescales.  The sensitivity of the implied timescales is generally low, meaning small variations in their values do not change the timescales significantly.  However, the relative sensitivity of the non-feature hyperparameters changes depending on type of feature used: e.g., with some features the number of microstates is most important while with others the timescales are not affected.  The VAMP variational scores, which have recently used to perform model selection on reversible MSMs have been shown to give results which are not in line with the variational principle underlying them. 

Given this relatively complex picture of hyperparameter optimisation we recommend the following for optimising reversible MSMs: 

\begin{enumerate}
    \item A number of different features should be tried. They can be selected from a list, potentially informed by the VAMP scores developed for by Schrere et. al. \cite{scherer_variational_2019}. 
    \item Remaining hyperparameters should be randomly sampled from a suitable search space. Using a grid search method is inefficient and not easily modified to sample more hyperparameter if necessary. 
    \item Scoring models using bootstrapped values of sum of squared eigenvalues is preferable to using the VAMP2 variational score because the latter can lead to contradictory results. 
    \item If the MSM requires significant resources to estimate, Bayesian optimisation may be a useful tool.  This work used a Gaussian process surrogate model but other, more flexible models may perform better (see limitations discussion below). 
    \item Even if the the MSM does not require significant resources, Bayesian optimisation provides a principled way of checking the convergence the model.  We suggest reporting an optimized model and a set of models which make it clear that the behaviour of model observables is consistent with variation in the hyperparameters. 
    \item When scoring the models use as many eigenvectors as are resolvable. This can be estimated from a selection of models or all processes can be scored and saved in a database (as was performed here).  
    \item For model selection, it is not important to select a `Markovain' lag time although this should be attempted to be estimated from a selection of models.  
\end{enumerate}

There are a two main limitations of this study.  First the we have a performed our analysis on only one system which limits the applicability some of the specific results. For example, the sensitivity of the hyperparameters to the timescales should be not taken as generally applicable for other systems. Second, the fitting of the Gaussian process used in the sensitivity calculation and the Bayesian optimisation required a lot of ad-hoc preprocessing and its own model selection procedure.  More flexible models which still return estimates of input importance and of uncertainty, such as random forests or tree parzen estimators, may be more appropriate.   


\subsection{References}



%%%%%%%%%%%%%%%%%%%%%%%%%%%%%%%%%%%%%%%%%%%%%%%%%%%%%%%%%%%%%%%%%%%%%
%% The "Acknowledgement" section can be given in all manuscript
%% classes. This should be given within the "acknowledgement"
%% environment, which will make the correct section or running title.
%%%%%%%%%%%%%%%%%%%%%%%%%%%%%%%%%%%%%%%%%%%%%%%%%%%%%%%%%%%%%%%%%%%%%
\begin{acknowledgement}

Please use ``The authors thank \ldots'' rather than ``The
authors would like to thank \ldots''.


\end{acknowledgement}

%%%%%%%%%%%%%%%%%%%%%%%%%%%%%%%%%%%%%%%%%%%%%%%%%%%%%%%%%%%%%%%%%%%%%
%% The same is true for Supporting Information, which should use the
%% suppinfo environment.
%%%%%%%%%%%%%%%%%%%%%%%%%%%%%%%%%%%%%%%%%%%%%%%%%%%%%%%%%%%%%%%%%%%%%
\begin{suppinfo}


\end{suppinfo}

%%%%%%%%%%%%%%%%%%%%%%%%%%%%%%%%%%%%%%%%%%%%%%%%%%%%%%%%%%%%%%%%%%%%%
%% The appropriate \bibliography command should be placed here.
%% Notice that the class file automatically sets \bibliographystyle
%% and also names the section correctly.
%%%%%%%%%%%%%%%%%%%%%%%%%%%%%%%%%%%%%%%%%%%%%%%%%%%%%%%%%%%%%%%%%%%%%
\bibliography{references.bib}

\end{document}





% \begin{itemize}
    % \item Markov state models are a popular model for analysing MD data. 
    % \item They are able to provide a quantitative picture of the conformational dynamics of biomolecular systems. 
    % \item They have been used to study protein folding, ligand binding, peptide- and protein-protein association,  enzymatic reaction dynamics. 
    % \item They have also been used in adaptive sampling algorithms where statistical properties of the model are used to select conformations from which to seed more MD simulations to speed convergence. 
    % \item Like any any statistical model, a number of choices must be made when estimating an MSM. These include choice of estimation algorithm, convergence criteria for optimizing loss-functions; batching, sub-sampling and splitting data for compute resource management and estimating out of sample accuracy; data pre-processing e.g., image resizing, feature-scaling and warping; feature selection and egineering, de-correlating etc. 
    % % \item These choices affect the outcome of the model to varying to degrees but are not `learned' from the data via minimizing a loss function, in the way the parameters of the model are (e.g., neural network weights, MSM transition matrix elements). For this reason they are called `hyperparameters' of the model. 
    % \item There has been a lot of recent attention paid to the affects of hyperparameter selection in, for e.g.,  psychology, neuroscience and machine learning  where opaque methods of hyperparameter selection have lead to irrepreducible results. 
    % \item There are a host of different approaches to this problem, which differ according to whether explanatory power or predictive accuracy of the model are required. 
    % \item When statistical models are made for their explanatory power variants of sensitivity analysis are often used. SA entails estimating models with  several plausible sets of hyperparameters to see how they affect the results [ref].  Similarly, multiverse analysis uses a more thorough enumeration of potential hyperparameters [ref].  Specification curve analysis [ref] also uses a multiverse of results while going further to infer information from the distribution of results. 
    % \item In machine learning, where predictive power is often more important, hyperparameters can be chosen to optimize performance metrics of the model, e.g., out of sample accuracy. Hyperparameters can be randomly or uniformly selected, or even optimized using e.g., Bayesian optimisation.  
    % \item The essential task in MSM estimation is to choose hyperparameters for preprocessing MD data into a smaller number of basis  states for which the MSM can be estimated. 
    % \item traditionally these basis states are discrete states corresponding to small regions of configuration space of the protein. Recent work has focused on estimating fuzzy basis sates using deep learning approaches.  
    % \item Either way a number of hyperparameters must be chosen. The resulting basis states can be judged according to variational scores. 
    % \item For reversible MSMs the generalized matrix Rayleigh coefficient (GMRQ) was introduced as metric of optimising basis states.  This was later expanded to include non-reversible and non-stationary MSMs with the variational approach to Markov processes (VAMP). 
    % \item By varying hyperparameters to increase the variational score, the basis states can be made increasingly accurate. 
    % \item  However, most recent papers which utilise MSMs as their man analytic tool, do not report how hyperparameters were chosen.  
    % \item Where methods for hyperparameter selection were discussed VAMP scores were generally used to discriminate between a handful of different hyperparameters.  
    % \item Taking inspiration from the literature on sensitivity analysis and hyperparameter optimisation we investigate whether more extensive hyperparameter selection methods are needed or appropriate. 
    % \item We investigate how sensitive how MSM timescales are to changes in hyperparameters, demonstrate how an active learning approach might work to finding good quality hyperparameters might work and comment on the commonly used VAMP scores for optimizing basis states.
    % \item This work is structured as follows. Section~\ref{theory} covers the theory of Markov state models and of hyperparameter optimisation and search strategies; section~\ref{methods} covers the methods and materials used; section~\ref{results} discusses results using the fast folding protein BBA as an example; section~\ref{conclusion} concludes with recommendations for estimating MSMs. 
% To get a sense of the current practice for estimating MSMs a small survey of recent literature was conducted. Web Of Science[] was used to look for articles citing PyEMMA~\cite{schererPyEMMASoftwarePackage2015a}, Enspara~\cite{porter_enspara_2019} and MSMBuilder~\cite{beauchamp_msmbuilder2:_2011} and Deeptime~\cite{deeptime}, published since 2020 and 25 randomly selected for detailed investigation~\cite{tosstorff_study_2020, fernandez-quintero_mutation_2021, kahler_sodium-induced_2020, paul_thermodynamics_2021, quoika_implementation_2021, liu_misfolding_2020, tian_deciphering_2020, hempel_molecular_2021, koulgi_structural_2021., sharma_comparative_2020, mckiernan_dynamical_2020, dutta_distinct_2022, zhou_molecular_2021, fernandez-quintero_cdr_2022, song_modulation_2021, sadiq_multiscale_2021, ibrahim_dynamics_2022, linker_polarapolar_2022, hu_discovery_2022, cannariato_prediction_2022, jones_determining_2021, zhu_critical_2021, zhu_critical_2021, bergh_markov_2021, pantsar_decisive_2022, grabski_molecular_2021}. Articles looking at purely methodological questions were excluded. Three questions were asked: 
% \begin{itemize}
%     \item Were sensitivity analyses performed? i.e., were the sensitivity of observables tested with respect to the model hyperparameters?
%     \item How did the authors select the hyperparameters? e.g., using VAMP score? 
%     \item Did the authors perform a validation of the selected model using implied timescales and/or a Chapman-Kolmogorov test? 
% \end{itemize}

% Only one of the studies presented a sensitivity analysis, (this analysis also served as a hyperparameters selection technique)~\cite{bergh_markov_2021}. The the majority (15, \SI{60}{\percent}) of studies did not discuss any hyperparameter search techniques and of those that did,  the majority~\cite{paul_thermodynamics_2021, koulgi_structural_2021, sharma_comparative_2020, dutta_distinct_2022, zhou_molecular_2021, jones_determining_2021, zhu_critical_2021, grabski_molecular_2021} used VAMP scores (8, \SI{89}{\percent}), while one article used the elbow method with important observables \cite{bergh_markov_2021} as their objective function. Only a minority~\cite{quoika_implementation_2021, hempel_molecular_2021, song_modulation_2021, ibrahim_dynamics_2022} failed to give evidence of validation.  

% It can be concluded that hyperparameter optimisation is popular but not universal and sensitivity analysis is either a) not performed or b) not thought important (either by journals or by the authors) enough not to include in the journal article. 
% \end{itemize}





% \begin{itemize}
%     \item The variational theorem applied to the transfer operator $\mathcal{T}(\tau)$, implies that the sum eigenvalues of the $\mathbf{T}$ in some arbitrary basis, will always be less than the sum of the true eigenvalues. 
%     \item Thus, one is free to choose a basis which will increase the sum of the eigenvalues. The associated timescales and eigenvectors will become closer to the true timescales and eigenvectors. 
%     \item In practice we restrict the score to pertain to the $k$ slowest processes, where $k=2 - \simeq 10$. We write the score as $S(\bm{\theta}; k) = \sum_{i=1}^{k}\lambda_{i}$. The functional dependence on $\bm{\theta}$ highlights the fact that the score is assessing the accuracy of the basis states. 
%     \item A general method of finding the most accurate basis states would be to vary the elements of $\bm{\theta}$ until a sufficiently large value of $S$ is found.  However, as pointed out in [mcgibbon] this would favour basis states which fit to noisy fluctuations in the data and do not represent the most accurate basis states, i.e., they would over-fit to the data at hand.
%     \item There are two common techniques to avoid over-fitting. First is the Bootstrap [ref bootstrap] and Cross-validation. The approach  taken by [Noe] and [Pande] is to use cross-validation. 
%     \item The cross-validated estimator of $S$ first estimates the eigenvectors on half of the discretized MD data (the matrix of eigenvectors is given by $\mathbf{U}^{i}$, where the $i$ denotes the $i$th training cross-validation split) the count ($\mathbf{C}_{01}^{-i}$, where $-i$ denotes the complementary test split to $i$) and population ($\mathbf{C}_{00}^{-i}$) matrices are estimated on the remaining half of the data. This is repeated $N$ times with the score being given by: 
%     \begin{equation}
%         GMRQ(\bm{\theta}; k) = \frac{1}{N}\sum_{i}^{N} \operatorname{Tr}\left[(\mathbf{U}^{iT}\mathbf{C}_{01}^{-i}\mathbf{U}^{i})(\mathbf{U}^{iT}\mathbf{C}_{00}^{-i}\mathbf{U}^{i})^{-1}\right]
%     \end{equation}\label{eqn:gmrq_cv_def}
%     \item In other words, this tests how well the training eigenvectors, diagonalize the test count and population matrices. 
%     \item Because the eigenvectors are estimated from a reversible transition matrix they are not consistent with the count matrix.  To account for this, a symmetrized count matrix is used: $\mathbf{C}_{01}^{\mathrm{rev}} = \mathrm{T}^{\mathrm{rev}}\cdot \bm{\Pi}^{\mathrm{rev}}$.  Where $\Pi$ is a diagonal matrix with the elements of $\pi$ on its diagonal. 
%     \item The VAMP scores follow the same principle as the GMRQ but with a distinction drawn between populations at time $t$ and at time $t+\tau$. I.e., $\mathbf{C}_{00} \neq \mathbf{C}_{11}$
%     \item this is to allow the possibility of non-stationary and non-reversible models.
%     \item Instead of eigenvectors, left and right singular vectors, $\mathbf{U}$ and $\mathbf{V}$ of the transition matrix are used (the cross-validation notation is dropped for clarity): 

%     \item When the count and population matrices and the singular vectors are all estimated from the same data, this amounts to the sum of the singular vectors raised to the power of $r$.  
%     \item When $\mathbf{C}_{00} = \mathbf{C}_{11}$ and $\mathbf{C}_{01}$ symmetric, then this expression with $r=1$ should be equivalent to the GMRQ.  i.e., the singular values should equal the eigenvalues.  
%     \item With $r=2$ this expression measures the kinetic variance~\cite{noeKineticDistanceKinetic2015} captured by the basis sets. 
%     \item Both scores truncate the eigenvector/singular vectors to the first $k$ components to restrict the score to just those processes. 
%     \item Both formulae can also be used with bootstrapping where the train and test data are the same but $N$ datasets are generated by sampling with replacement from the pool of available MD trajectories.   
%     \item The VAMP-r scores have also been adapted to score the features only [ref]. 
% \end{itemize}